\onecolumn

\chapter{Sobre la preparaci�n de este libro}\label{AnexoPreparacion}

Puesto que este es un libro libre, tienes a tu disposici�n no solo la versi�n lista para leer (ya sea en un archivo PDF o en una copia impresa), sino tambi�n el formato editable sobre el que yo he trabajado para crearlo. En caso de que desees editarlo o aprovecharlo de alg�n modo, aqu� tienes algo de informaci�n adicional acerca de c�mo he creado este libro que quiz�s te resulte de ayuda. Todo el \emph{software} que he empleado es libre, por lo que puedes descargarlo de Internet y usarlo sin restricciones (aunque algunos de estos programas solo funcionan sobre el sistema operativo Windows, que no lo es).

Para preparar el libro, he utilizado el sistema de composici�n de textos \LaTeX. Las fuentes en formato \LaTeX, as� como todas las im�genes y ficheros adicionales necesarios, las puedes descargar del siguiente repositorio SVN, usando un cliente SVN cualquiera.

\begin{quote}
\texttt{https://svn.forge.osor.eu/svn/sextante/trunk/docs/LaTeX/es/LibroSIG}
\end{quote}

He empleado TeXnic Center como entorno \LaTeX. Encontrar�s junto a esas fuentes un fichero llamado \texttt{Libro\_SIG.tcp} con un proyecto de TeXnic Center que puedes abrir directamente.

Las figuras, siempre que esto es posible, est�n en formato vectorial. Las he creado usando Inkscape y est�n almacenadas en formato SVG de Inkscape. Las restantes figuras est�n almacenadas como im�genes en formato PNG. En aquellas figuras que muestran mapas o resultados de an�lisis dentro de un SIG, los an�lisis y operaciones correspondientes, as� como la creaci�n de la representaci�n en s�, han sido realizados empleando los SIG SAGA y gvSIG.

La bibliograf�a en formato BibTeX la encontrar�s en el fichero \texttt{Libro\_SIG.bib}. Para gestionarla te recomiendo el excelente JabRef.

Quiero agradecer a todos los desarrolladores de estas aplicaciones su fant�stico trabajo. Sin su \emph{software}, sacar adelante este libro habr�a sido, sin duda, mucho m�s dif�cil y costoso.
