\pagestyle{empty}
\graphicspath{{Fundamentos/}}
\part{Los fundamentos}\label{Fundamentos}

\sffamily

\vspace*{4cm}

En esta parte comenzar�s a conocer qu� es un SIG, el porqu� de su existencia, su utilidad, y los distintos componentes en que podemos dividirlos, y que ser�n estudiados de forma separada a lo largo de todo el libro. Adem�s de esto, se presentan en esta parte algunos conceptos relativos a ciencias afines como la cartograf�a o la geodesia, que son imprescindibles para poder comprender en profundidad los SIG y sus distintas facetas.

\begin{itemize}

\item El cap�tulo \ref{Introduccion_fundamentos} presenta el entorno de los SIG, mostrando al lector el contenido gen�rico sobre el que trata no solo esta parte, sino el libro al completo. Se describen las ideas fundamentales sobre SIG y los elementos que lo forman.

\item El cap�tulo \ref{Historia} recorre la breve pero intensa historia de los SIG, desde su origen hasta nuestros d�as.

\item En el cap�tulo \ref{Fundamentos_cartograficos} se resumen los conceptos cartogr�ficos y geod�sicos b�sicos, imprescindibles para el d�a a d�a del trabajo con un SIG.

\end{itemize}

\rmfamily
%\twocolumn

\include{Fundamentos/Introduccion_fundamentos/Introduccion_fundamentos}
\include{Fundamentos/Historia/Historia}
\include{Fundamentos/Fundamentos_cartograficos/Fundamentos_cartograficos}
