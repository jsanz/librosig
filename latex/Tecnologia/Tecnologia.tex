\graphicspath{{Tecnologia/}}
\part{La tecnolog�a}\label{Tecnologia}

\pagestyle{empty}

\onecolumn

\sffamily

\vspace*{4cm}

En esta parte trataremos los programas inform�ticos en s� (el \emph{software}) y otros componentes tecnol�gicos que dan forma a los SIG.

\begin{itemize}
\item El cap�tulo \ref{Introduccion_tecnologia} presenta una clasificaci�n de los distintos tipos de elementos que encontramos actualmente en el campo de los SIG. Estos ser�n los elementos que se describan con detalle en los cap�tulos que le siguen.

\item El cap�tulo \ref{SIGs_escritorio} desarrolla las herramientas de escritorio o aplicaciones independientes, que tradicionalmente se identifican con el concepto cl�sico de SIG.

\item El cap�tulo \ref{Servidores_y_clientes_remotos} trata sobre todos aquellos elementos, sea en el lado del servidor o del cliente, que sirven para el empleo de datos remotos. Se incluye aqu� todo lo relativo a servicios cartogr�ficos en Web (\emph{Web mapping}) y desarrollos asociados.

\item El cap�tulo \ref{Otros_tecnologia} desarrolla los elementos del SIG m�vil, una de las ramas del SIG con mayor proyecci�n en la actualidad, y que est� aportando una verdadera revoluci�n en este campo.


\end{itemize}

\rmfamily

\include{Tecnologia/Introduccion_tecnologia/Introduccion_tecnologia}
\include{Tecnologia/SIGs_escritorio/SIGs_escritorio}
\include{Tecnologia/Servidores_y_clientes_remotos/Servidores_y_clientes_remotos}
\include{Tecnologia/Otros_tecnologia/Otros_tecnologia}

