\chapter{Ciencias sociales}\label{Ciencias_sociales}

\begin{keypoints}
�Qu� beneficios aporta el uso de SIG en las ciencias sociales?$\bullet$ �Como utiliza un SIG un historiador? $\bullet$ �Y un arque�logo? $\bullet$ �C�mo se modeliza en un SIG la propagaci�n de una plaga? $\bullet$ �C�mo se utiliza un SIG en un estudio criminol�gico para detectar patrones y proponer actuaciones?
\end{keypoints}

\bigskip

\begin{intro}
Aunque los SIG nacieron como herramientas vinculadas al estudio del medio natural, muchas otras disciplinas han sabido sacar tanto o m�s partido de ellos que las propias ciencias del medio. Entre ellas, las ciencias sociales hacen un uso muy diverso y rico de los SIG, pues todas ellas incluyen en alg�n punto alguna componente geogr�fica en cuyo an�lisis los SIG pueden ser herramientas de primer orden.

En este cap�tulo estudiaremos esos usos y veremos la versatilidad de los SIG para aplicaciones de toda �ndole, desde sencillos modelos demogr�ficos a aplicaciones en arqueolog�a, donde las necesidades del estudio del tiempo como variable fundamental hacen necesarios elementos de un SIG muy distintos.
\end{intro}

%\bibliographystyle{unsrt}
%\bibliography{../../Libro_SIG}