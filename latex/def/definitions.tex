\usepackage{latexsym}
\usepackage{amssymb}
\usepackage{amsmath}
\usepackage[pdftex]{graphicx}
\usepackage[spanish]{babel}
\usepackage[latin1]{inputenc}
\usepackage[switch,pagewise]{def/lineno}
\usepackage{makeidx}
\makeindex
\usepackage{tabularx}
\usepackage{array}
\usepackage{hyperref}
\usepackage{xspace}
\usepackage{color}
\usepackage{booktabs}
\usepackage{fancyhdr}
\usepackage{geometry}
\usepackage{fixltx2e}

\geometry{verbose,dvips,paperwidth=17cm,paperheight=23.5cm,tmargin=1.8cm,bmargin=1.6cm, left=1.8cm, right=1.2cm}
\newcommand{\mycolumnwidth}{\textwidth}

%\linenumbers

\renewcommand{\chaptermark}[1]{\markboth{#1}{}}

\fancyhf{}
\fancyhead[LE,RO]{\textsc{\thepage}}
\fancyhead[RE]{\textsc{Sistemas de Informaci�n Geogr�fica}}
\fancyhead[LO]{\textsc{\leftmark}}
\fancypagestyle{plain}{%
\fancyhead{} 
\renewcommand{\headrulewidth}{0pt}
}

\numberwithin{equation}{section}
\renewcommand{\theequation}{\thesection.\arabic{equation}}

\makeatletter
\def\thickhrulefill{\leavevmode \leaders \hrule height 1ex \hfill \kern \z@}
\def\@makechapterhead#1{%
  \vspace*{10\p@}%
  {\parindent \z@
    {\raggedleft \reset@font
      \scshape \@chapapp{} \thechapter\par\nobreak}%
    \par\nobreak
    \vspace*{20\p@}
    \interlinepenalty\@M
    {\raggedright \Large \bfseries #1}%
    \par\nobreak
     \hrulefill
     \par\nobreak
    \vskip 20\p@
  }}
\def\@makeschapterhead#1{%
  \vspace*{10\p@}%
  {\parindent \z@
    {\raggedleft \reset@font
      \scshape \vphantom{\@chapapp{} \thechapter}\par\nobreak}%
    \par\nobreak
    \vspace*{20\p@}
    \interlinepenalty\@M
    {\raggedright \Large \bfseries #1}%
    \par\nobreak
    \hrulefill
    \par\nobreak
    \vskip 20\p@
  }}

%\makeindex

\renewcommand{\v}[1]{\ensuremath{\overline{\mathbf{#1}}}}
\newcommand{\chapterauthor}[1]{\begin{center}\sffamily{#1}\end{center}\medskip}
\newcommand{\degree}{\ensuremath{^\circ}}
\newenvironment{intro}{\sffamily \small}{\normalsize\par\nobreak\noindent \rule{\textwidth}{0.5pt}}
\newenvironment{keypoints}{ \small\sffamily\bfseries}{\small}


%Colores
\definecolor{urlcolor}{rgb}{0,0,0.5}
\definecolor{granate}{rgb}{0.5,0.0,0.0}

%Comandos
\newcommand{\extr}[1]{\emph{#1}\xspace}
\newcommand{\footurl}[1]{\footnote{\url{#1}\xspace}}
%\newcommand{\soft}[1]{#1\xspace}

\newcommand{\SL}{software libre\xspace}

%%Configuraci�n de hyperref, SE RECOMIENDA DEJAR SIEMPRE AL FINAL DEL PRE�MBULO
\hypersetup{
    pdftitle=Sistemas de Informaci�n Geogr�fica,
    pdfstartview=FitH,
    pdfproducer=PDFLaTeX,
    naturalnames,
    plainpages,
    colorlinks,
    linkcolor=black,
    urlcolor=urlcolor,
    citecolor=granate,
    }